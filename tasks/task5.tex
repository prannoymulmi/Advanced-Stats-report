\chapter{Assignment 5: Sufficient Statistics}
We have a probability density function $f_x(x,\theta) = \frac{1}{6! \theta^5}x^6e^{\frac{-x}{\theta}}$ which depends on the parameter $\theta \in \mathbb{R}^+$ and we want to find a sufficient statistic $Y=u(x_1, x_2,..x_n)$. In order to calculate the sufficient statistics, we would firstly have to rewrite the given probability function of the random sample $(X_1, X_2, ..., X_n)$ into its joint density function which is as follows:
\begin{equation} \label{eqn:joint_pdf_task_5}
        f(x|\theta) = \prod_{i=1}^{n} \frac{1}{6! \theta^5}x^6e^{\frac{-x_i}{\theta}} = \frac{1}{6!^n \theta^{5n}}x_i^{6n}e^{\frac{-1}{\theta} \sum_{i=1}^{n}x_i} 
\end{equation}
Now we can apply the Neyman's factorization theorem where our statistic Y is a sufficient statistic of the parameter $\theta$ when we can find two non-negative functions g and h given $f(x_1;\theta)f(x_2;\theta)...f(x_n;\theta)=h(x) g_{\theta}(Y(x))$ and the function h does not depend upon the parameter $\theta$ (see \cite{hogg:2005}, pg. 436). Using the theorem we can arrange the joint density function that we defined in equation \ref{eqn:joint_pdf_task_5} in such a way that we have the two functions h and g which is as follows:
\begin{equation} \label{eqn:non_neg_function_task_5}
   h(x) g_{\theta}(Y(x)) = \frac{x_i^{6n}}{6!^n} \times \frac{e^{\frac{-1}{\theta} \sum_{i=1}^{n}x_i}}{\theta^{5n}} 
\end{equation}\newline\newline
We can see in equation \ref{eqn:non_neg_function_task_5} that the function is now divided into two parts, one which is dependent on the parameter $\theta$ g and the function which does not depend on $\theta$ but only on x. As we can see both the functions are non negative and the function h is independent of $\theta$,now we can say that $\sum_{i=1}^{n}x_i$ is the sufficient statistic for our parameter. 