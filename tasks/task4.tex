\chapter{Assignment 4: Hypothesis Test}
It is given that we have a sample data from a long running production of 1000 quality hammers with an averaged weight of $\mu =$ 971g and a standard deviation $s= $ 15.2g. For this given data we can model the given data using a normal distribution. Following the central limit theorem see (\cite{hogg:2005}, pg.355), a random sample from size n which has a definite mean $\mu $ and variance $s^2$ would converge towards a Gaussian or a normal distribution. Assumptions made for which the models hold are as follows:
\begin{enumerate}
    \item The random variable for the sample are located mostly around the mean and outliers are very rare.
    \item The random variables are independent of one another.
    \item The sample observation should be of a continious nature
\end{enumerate}
As the model is assumed to be a normally distributed the parameters of them are the mean and its variance (\cite{Iubh:2021}, pg. 75) $\mathcal{N}(\mu, s^2) = \mathcal{N}(971, 231.04)$.\newline
Given, are the weights of the newly produced hammers, with sample size n= 10:
\begin{center}
    987, 966, 955, 977, 981, 967, 975, 980, 953, 972
\end{center} 
We can calculate the sample mean $\mu = \frac{\sum_{i=1}^{n}x_i}{N}=971.3$, the sample variance $s^2=\frac{\sum_{i=1}^{n}(x_i-\mu)^2}{N-1} = 123.34$ and sample standard deviation $s=\sqrt{s^2}=11.10$ from the given new production samples.
Now, for the random sample of the weights of new hammer production factory $W = (X_1, X_2, ..., X_{10})$ we can formulate the following hypothesis in accordance to the already existing hammer, where $\mu$ is the population mean for production line with n=1000:\newline
\begin{equation}
    \begin{split}
      H_0 : \mu_0 = \mu& \text{ (Null hypothesis)}\\
      H_a: \mu_0 \neq \mu& \text{ (Alternate hypothesis)}
    \end{split}
\end{equation}
This hypothesis is chosen because we want to check if there exists a similarity between the production for the weight of the hammers of the two production lines. When we assume that the mean of the new production site is equal to the existing one we propose that the weights might produce a more constant weight if we do not reject the null hypothesis. Also, when we reject the null hypothesis it can be said that the weights are not consistent.\newline\newline
Furthermore, in the given case we do not know the neither the population mean nor the population variances, we rather only have the information about two sample means and their sample variances. Therefore, in this case a welch's test (see \cite{hogg:2005}, pg. 200) deems to be suitable to carry out the test statistics. We can define out test statistic as T and the critical region as C where $P(|T| > C) =\frac{\alpha}{2}$ \label{eqn:alpha_2_rr}:
\begin{equation}
    \begin{split}
      |T| > C_{\frac{\alpha}{2}, DOF}& \text{ (Reject $H_0$)}\\
      |T| < C_{\frac{\alpha}{2}, DOF}& \text{ (Do not reject $H_0$)}
    \end{split}
\end{equation}
The critical region or the cutoff can be found out if we look for the significance level $\alpha$ for the corresponding value for the degree of freedom in the T-Distribution table, in our case for a two tailed distribution. A two tailed distribution is chosen for this test because we would like to account for both the positive and the negative sides of the rejection regions giving us more evidence of the similarity between the two samples. This is also the reason why the significance level or the error type I (see \ref{eqn:alpha_2_rr}) has to be divided into two. \newline \newline
An error I probability, which is the error of rejecting the null hypothesis when $H_0$ is true $\alpha$ of 0.05 or 5\% sounds plausible for this problem. The 5\% error is a widely used value for the $\alpha$. In this experiment the precision for error does not have to be small and an error for this range can be tolerated. Now, we can calculate the degrees of freedom for this test which can be approximated using the harmonic mean (see \cite{Iubh:2021}, pg. 200) which is as follows:
\begin{enumerate}
    \item $d.o.f = \frac{2}{\frac{1}{n_1}+\frac{1}{n_2}}$ where $n_1$= 1000 and $n_2$ = 10
    \item $d.o.f = \frac{2}{\frac{1}{1000}+\frac{1}{10}}$
    \item $d.o.f = 19.8$
    \item $d.o.f = 20$ ( rounded Up)
\end{enumerate}
With the d.o.f we can look up in the T-Table distribution (see \cite{t-table}) to find out the critical region for the $\alpha$ value of 0.05 of the two tailed distribution for 20 degrees of freedom which is 2.086 and we can rewrite the probability of the rejection region as $P(|T| > 2.086) = \frac{\alpha}{2}$. Firstly, we calculate the $T_{obv}$ from the test statistic using the welch's test. The calculation is as follows:
\begin{enumerate}
    \item $T_{obv} = \frac{<X_1> - <X_2>}{\sqrt{\frac{S_1^2}{n_1} + \frac{S_2^2}{n_2}}}$ where $X_1 = 971, X_2 = 971.3, S_1 = 15.2 \text{ and } S_2 = 11.10$ 
     \item $T_{obv} = \frac{971 - 971.3}{\sqrt{\frac{231.04}{1000} + \frac{123.34}{10}}}$
      \item $T_{obv} = - 0.084$
      \item $|T_{obv}| = 0.084$
\end{enumerate}
Similarly, we can also calculate the p-value for the given test which is given as $P(T > |T_{obv}| | H_0) = 2P(T > 0.084 | H_0) = 0.933 $ (Calculated using wolfram alpha). Lastly, we can now conclude and say that we do not reject our null hypothesis because the the calculated p-value(0.933)  $> \alpha$ (0.05), which gives us a hint that these two samples are not statistically significant or in other words very similar and produce consistent results.